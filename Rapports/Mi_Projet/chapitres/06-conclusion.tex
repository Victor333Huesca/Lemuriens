\chapter*{Conclusion}
\addcontentsline{toc}{chapter}{Conclusion}
\markboth{Conclusion}{Conclusion}
\label{sec:conclusion}

% Dégelasse
%\linespread{0.7}
\setlength{\parskip}{0.7pt}

\section*{Difficultés rencontrées}
    \paragraph{}Comme toujours, un projet s'accompagne de difficultés plus ou moins prévues. Tout d'abord, nous avons eu beaucoup de mal à trouver des informations sur notre matériel. La documentation de Libelium est très éparpillée, son API n'aidant pas puisque n'étant pas très compréhensible comme décrit plus haut. Bien que nous avions le rapport du groupe précédent pour nous aider à appréhender le matériel, celui-ci n'a pas pu répondre à toutes nos interrogations. Ceci souligne bien l'importance des documentations dans tout projet.
    
    \paragraph{}Passées plusieurs séances pour rassembler les informations importantes, nous avons eu une nouvelle difficulté pour faire fonctionner l'IDE. Alors que le programme installé sur les Waspmotes devait afficher sur l'écran des caractères (<< Hello World >> par exemple), il n'affichait que des points d'interrogations et autres caractères spéciaux. Le souci provenait de la fréquence de transmission USB entre l'ordinateur de développement et la Waspmote. Nous avons trouvé cette information après avoir essayé de nombreuses actions et consulté la documentation et le forum de Libelium.

    \paragraph{}Enfin, le dernier problème principal que nous avons rencontré est la non-comptabilité entre nos Waspmotes et la version de l'IDE (et par extension de l'API). En effet, entre la fin du projet du précédent groupe et le début du notre (environ un an et demi), de nombreuses versions de l'API et du matériel sont sorties et les dernières ont cassé la compatibilité avec nos Waspmotes. Comme dit précédemment, la documentation et les ressources de Libelium sont un peu éparpillées et nous avons mis un certain temps pour nous rendre compte de ce souci de version. \footnote{D'autant plus que tout n'a pas été rendu obsolète, de nombreuses fonctions sont encore compatibles. Nous laissant ainsi croire à des erreurs de manipulation de notre part.}
    
    % On est en plein dedans donc il ne nous à pas vraiment encore apporté quoi que ce soit, en revanche il nous a permit de mieux appréhender notre environnement de travail. De plus nous avons pu avoir une idée plus précise de l'organisation nécessaire pour le semestre à venir notamment au niveau du déploiement.
    
\section*{Perspectives}
    \paragraph{}Cette première partie nous a permis de comprendre et d'appréhender les mécanismes de développement propres à l'\emph{IoT} et plus particulièrement à l'environnement Libelium. De plus, cette phase de recherche touchant à sa fin, nous avons dorénavant à l'esprit l'organisation nécessaire à la mise en place du scénario de déploiement pour le prochain semestre.
    
    \paragraph{}La seconde partie sera logiquement orientée sur l'installation et la programmation du matériel. Elle nous permettra de concrétiser les acquis de la phase de recherche ainsi que de mettre en \oe uvre les prototypes de tests que nous avons pu développer.\\
    De plus, avec l'arrivée de quatre nouveaux étudiants de L3 informatique dans le projet, il sera possible de se repartir efficacement les tâches de développement et de déploiement apportant une expérience plus riche pour l'ensemble du groupe. Les deux tâches principales seront d'une part la mise en place du \emph{backend} (correspondant à l'aspect << Réseau et Capteurs >>), et d'autre part l'installation et le développement du \emph{frontend} (correspondant à l'aspect << Serveur, Web et Interface >>).